\chapter{Applied Model}
\label{ch:applied-model}

In this chapter we cover the model implementations of CBT and LKT-FM. Both models are implemented using the Python programming language version 3.6 and multiple libraries, such as \textit{Numpy} \cite{10.1109/MCSE.2011.37} and \textit{Scipy} \cite{scipy} for matrix computation and \textit{Scikit-learn} \cite{10.5555/1953048.2078195} to compute K-Means. The experiments are built upon the \textit{RecSys@Polimi framework} by Maurizio Ferrari Dacrema \cite{recsys-polimi-framework}, which offers dataset readers, baseline recommender systems, similarity computation and evaluation metrics.\\
The first section describes the implementation of hyperparameter tuning for the experiments.\\
The second section covers the implementations of CBT codebook construction and transfer algorithms.\\
The third section covers the LKT-FM variant of codebook transfer and describes the implementation of matrix factorization for codebook construction and of factorization machines for codebook transfer.



\section{Hyperparameter Tuning}

To perform experiments, hyperparameters for each model are optimized using a wrapper of Scikit-optimize \cite{10.5281/zenodo.1170575} for bayesian optimization.\\
Bayesian optimization \cite{bayesian-optimization} is a method to optimize black-box functions that take long time to evaluate by building a surrogate for the objective and quantifying the uncertainty in that surrogate using a bayesian machine learning technique called \textit{Gaussian process regression}. It then uses an acquisition function defined from the surrogate to decide where to sample in the provided hyperparameter ranges.\\
The wrapper accepts as input a set of hyperparameter ranges, an amount of cases, which is the number of iterations of training and evaluation to perform, an amount of random starts, which is the numbers of times the optimization process initializes the hyperparameters randomly within the provided range, and a target evaluation metric, used to perform comparison between iterations.\\
Each training and evaluation iteration during hyperparameter tuning is performed using a train set and a validation set, which has the same properties as the final test set, while they are disjoint.\\
Further details about the hyperparameters and other tuning variables are covered in the single experiments in \autoref{ch:experiments}.



\section{CBT}


\subsection{Codebook Construction}
\label{ss:cbt-codebook-construction}

Since ONMTF does not monotically decrease the function, we opt for an early stopping approach to ensure a suitable amount of iterations.\\
The code for ONMFT is as follows:
\begin{minted}[breaklines]{python}
import numpy as np
from numpy.linalg import multi_dot
from sklearn.cluster import KMeans

def initialize(self, X_s, k, l):

    # Source user-rating matrix
    self.X_s = X_s
    self.m, self.n = X_s.shape
    # Amount of codebook user clusters
    self.k = k
    # Amount of codebook item clusters
    self.l = l
    # ONMTF factorization matrices
    self.F = None
    self.G = None
    self.S = None

    # Initialize the factorization matrices
    _initialize_clustering()

def update():
    _update_G()
    _update_F()
    _update_S()
    return _loss()

def _initialize_clustering():
    """
    Initialize F, G, and S latent matrices according to Ding et al. 2006:
    1. G is obtained via k means clustering of columns of X_s. G = G + 0.2
    2. F is obtained via k means clustering of rows of X_s. F = F + 0.2
    3. S is obtained via S = F.T @ X_s @ G
    """
    self.G = _cluster(self.X_s.T, self.l)
    self.G += 0.2
    self.F = _cluster(self.X_s, self.k)
    self.F += 0.2
    self.S = multi_dot([self.F.T, self.X_s, self.G])

def _cluster(X, k):
    kmeans = KMeans(n_clusters=k, random_state=0).fit(X)
    labels = set(kmeans.labels_)
    labeled_features = kmeans.labels_
    return np.array([np.multiply([i == k for i in labeled_features], 1) for k in labels]).T.astype(np.float64)

def _update_G():
    enum = multi_dot([self.X_s.T, self.F, self.S])
    denom = multi_dot([self.G, self.G.T, self.X_s.T, self.F, self.S])
    self.G *= np.nan_to_num(np.sqrt(enum / denom))

def _update_F():
    enum = multi_dot([self.X_s, self.G, self.S.T])
    denom = multi_dot([self.F, self.F.T, self.X_s, self.G, self.S.T])
    self.F *= np.nan_to_num(np.sqrt(enum / denom))

def _update_S():
    enum = multi_dot([self.F.T, self.X_s, self.G])
    denom = multi_dot([self.F.T, self.F, self.S, self.G.T, self.G])
    self.S *= np.nan_to_num(np.sqrt(enum / denom))

def _loss():
    FSGt = multi_dot([self.F, self.S, self.G.T])
    loss = np.linalg.norm(self.X_s - FSGt)
    return loss
\end{minted}
where the \textit{update} function is called by the framework with an early stopping approach.\\
The matrix binarization and codebook construction is then performed as follows:
\begin{minted}[breaklines]{python}
def codebook_construction(X_s, F, G):

    F = _binarize_matrix(F)
    G = _binarize_matrix(G)

    sum_vector = np.ones(X.shape)
    enum = multi_dot([F.T, X_s, G])
    denom = multi_dot([F.T, sum_vector, G])

    B = np.nan_to_num(enum / denom)
    return B

def _binarize_matrix(X):
    """
    Binarize input matrix with 0 and 1
    """
    for i in range(X.shape[0]):
        # Find largest score in the row
        max_score = sorted(list(X[i, :])).pop()
        # Replace all other scores into 0
        X[i, :] = np.where(X[i, :] == max_score, X[i, :], 0)
        # Replace largest score with 1
        X[i, :] = np.where(X[i, :] == 0, X[i, :], 1)
    return X_copy
\end{minted}


\subsection{Codebook Transfer}

The codebook transfer technique provided in \autoref{ss:codebook-transfer} monotically decreases the loss function. Thus ensures that it is possible to stop the algorithm on the iteration in which the loss function returns a value equal to the one of the previous iteration.\\
Since the algorithm always converges to a local minimum, it is possible to execute it several times to further minimize the loss function in alternative local minima.\\
The code for codebook transfer is as follows:
\begin{minted}[breaklines]{python}
import numpy as np
from numpy.linalg import multi_dot

def initialize(X_t, B, transfer_attempts, maximum_fill_iterations):

    # Target user-rating matrix to be filled
    self.X_t = X_t
    self.p, self.q = X_t.shape
    # Codebook
    self.B = B
    self.k, self.l = B.shape
    # Amount of searches for different local minima to perform
    self.transfer_attempts = transfer_attempts
    # Maximum iterations to perform before the attempt is forced to stop
    self.maximum_fill_iterations = maximum_fill_iterations
    # Masking matrix
    self.W = None
    self.W_flipped = None
    # Factorization matrices
    self.U = None
    self.V = None
    self.U_best = None
    self.V_best = None
    # Best loss until now
    self.loss_best = np.inf

    # Initialize the masking matrix
    _generate_W()

    # Search local minima
    for attempt in range(self.transfer_attempts):

        _initialize_U()
        _initialize_V()
        iteration_loss_best = np.inf

        for i in range(self.maximum_fill_iterations):
            _update_U()
            _update_V()
            # Remove the filled values from the generated target matrix and compare it to the original one
            loss = np.linalg.norm((self.X_t - multi_dot([self.U, self.B, self.V.T])) * self.W)
            # The loss function is monotone, so stop if we have reached the local minimum
            if loss == iteration_loss_best:
                break
            iteration_loss_best = loss

        # Generate multiple local minima and keep the best one
        if iteration_loss_best < self.loss_best:
            self.loss_best = iteration_loss_best
            self.U_best = self.U
            self.V_best = self.V

def fill_matrix():
    prediction = self.W_flipped * multi_dot([self.U_best, self.B, self.V_best.T])
    X_t_filled = self.W * self.URM_target_train + prediction
    return X_t_filled

def _generate_W():
    self.W = np.zeros((self.p, self.q))
    self.W = np.divide(self.URM_target_train, self.URM_target_train)
    self.W = np.nan_to_num(self.W)
    self.W_flipped = 1 - self.W

def _initialize_U():
    self.U = np.zeros((self.p, self.k))

def _initialize_V():
    self.V = np.zeros((self.q, self.l))
    for i in range(self.q):
        j = np.random.randint(self.l)
        self.V[i, j] = 1.0

def _update_U():
    BV_t = np.dot(self.B, self.V.T)
    for i in range(self.p):
        loss = np.ndarray(self.k)
        for j in range(self.k):
            loss[j] = np.linalg.norm((self.X_t[i, :] - BV_t[j, :]) * self.W[i, :])
        j = np.argmin(loss)
        self.U[i, :] = 0.0
        self.U[i, j] = 1.0

def _update_V():
    UB = np.dot(self.U, self.B)
    for i in range(self.q):
        loss = np.ndarray(self.l)
        for j in range(self.l):
            loss[j] = np.linalg.norm((self.X_t[:, i] - UB[:, j]) * self.W[:, i])
        j = np.argmin(loss)
        self.V[i, :] = 0.0
        self.V[i, j] = 1.0
\end{minted}



\section{LKT-FM}


\subsection{Codebook Construction}

As covered in \autoref{ch:analysed-models}, LKT-FM uses basic matrix factorization for source domain clustering, instead of ONMTF.\\
The code for matrix factorization is as follows:
\begin{minted}[breaklines]{python}
import numpy as np

def initialize(X_s, d, h, alpha, beta, iterations):

    # Source user-rating matrix
    self.X_s = X_s
    self.num_users, self.num_items = X_s.shape
    # Amount of user latent dimensions
    self.d = d
    # Amount of item latent dimensions
    self.h = h
    # Learning rate
    self.alpha = alpha
    # Regularization parameter
    self.beta = beta
    # Amount of iterations
    self.iterations = iterations
    # Matrix factorization latent matrices
    self.P_s = np.random.normal(scale=1./self.d, size=(self.num_users, self.d))
    self.Q_s = np.random.normal(scale=1./self.h, size=(self.num_items, self.h))
    # Matrix factorization biases
    self.b_u = np.zeros(self.num_users)
    self.b_i = np.zeros(self.num_items)
    self.b = np.mean(self.X_s[np.where(self.X_s != 0)])

    self.samples = [
        (i, j, self.X_s[i, j])
        for i in range(self.num_users)
        for j in range(self.num_items)
        if self.X_s[i, j] > 0
    ]

    for i in range(self.iterations):
        np.random.shuffle(self.samples)
        _sgd()

def _sgd():
    for i, j, r in self.samples:
        prediction = self.get_rating(i, j)
        e = (r - prediction)
        self.b_u[i] += self.alpha * (e - self.beta * self.b_u[i])
        self.b_i[j] += self.alpha * (e - self.beta * self.b_i[j])
        self.P_s[i, :] += self.alpha * (e * self.Q_s[j, :] - self.beta * self.P_s[i, :])
        self.Q_s[j, :] += self.alpha * (e * self.P_s[i, :] - self.beta * self.Q_s[j, :])
\end{minted}
The matrices are then binarized and used to build the codebook like described in \autoref{ss:cbt-codebook-construction}.


\subsection{Codebook Transfer}

While LKT-FM uses the same technique of CBT for target domain clustering, it uses factorization machines to expand the codebook into the target user-rating matrix, instead of direct expansion.\\
The implementation of LKT-FM codebook transfer uses \textit{pylibFM} \cite{pylibfm}, a Python wrapper for \textit{libFM} \cite{10.1145/2168752.2168771, libfm}.