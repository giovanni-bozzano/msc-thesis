\chapter{Conclusion}
\label{ch:conclusion}

In the previous chapters, we analyzed the state of the art of pattern transfer recommender systems. Moreover, we focused on its variant without overlap, currently known as codebook transfer. Codebook transfer, given a meaningful source domain, should be able to alleviate the data quantity and cold start problems of a target domain by filling the data gaps with knowledge extracted from the rating pattern of groups of users.\par
In our ranking experiments, we compared the original codebook transfer method and LKT-FM, an evolution of it, with multiple baselines, with various combinations of well-known source and target datasets. Each experiment includes comparisons with random, top popular items and $k$-nearest neighbors without knowledge transfer recommender systems. The results provide evidence that the performance of codebook transfer is inferior to the one of classic recommender systems, such as $k$-nearest neighbors without knowledge transfer and, in some cases, top popularity items recommendation.\\
Furthermore, by expanding on the previous experiments by Cremonesi and Quadrana on rating prediction, it is possible to say that codebook transfer can improve error metrics without transferring knowledge, due to its nature being very similar to that of SVD matrix factorization in the codebook transfer phase.\\
For this same reason, it is not suitable for the ranking task as, indeed, it mostly fills the target user-rating matrix with noise. To prove this, we applied various transformations to the source domain, including rating removal and random generation, and obtained very similar results to the ones obtained with the original source domain.\\
Results including LKT-FM also underline that the derived approach, which employs an alternative codebook expansion method using factorization machines, is outperformed by almost every baseline, having similar performance to the random recommender system.\par
To summarize, in this thesis we aim to expand on the outcome of previous evaluations with theoretical and empirical evidence that the original codebook transfer method and its derived approaches are not capable of providing solid knowledge transfer between non-overlapping domains.\\
Future research may be focused on providing alternative methods to extract meaningful knowledge from source domains.