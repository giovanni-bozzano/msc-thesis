\chapter*{Abstract}

With the ever increasing popularity of entertainment streaming services, social media, e-commerce websites etc., very large and diverse catalogs become usually hard to approach by users. Predicting the single customer's tastes and creating personalized lists of products has become a need for companies.\\
By aiming at alleviating this problem, recommender systems have recently become major players in the field of machine learning. These systems leverage the data of users and items of the catalog to extract information about each user's taste.\par
There are mainly two branches of recommender systems which differentiate by approach:
content-based recommender systems extract results based on item similarity, while collaborative filtering recommender systems take advantage of the user profiles and interactions history.\par
The quality of the results strictly depends on data quality and quantity. In particular, in practical applications, the amount of interactions has extremely low density compared to the whole dataset. That is, users tend to interact with or review a very small subset of the catalog. This issue is know as the data sparsity problem.\\
Over the years, many different knowledge transfer solutions to this problem have been proposed. The majority of these techniques relies on the overlap of users, items or both, between two dataset, source and target.\par
In this thesis we aim to analyse a particular set of knowledge transfer techniques based on rating pattern transfer, without any data overlapping between datasets. We show their formulation and performance with experiments on different datasets.